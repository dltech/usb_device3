% тут сейчас будут очевидные коментарии
\documentclass[12pt,a4paper]{article}
\usepackage[utf8]{inputenc}				% кодировка утф8
\usepackage[T2A]{fontenc}
\usepackage[russian]{babel}				% локализация
\usepackage{misccorr}					% пакет с дополнительными настройками для соответствия правилам отечественной полиграфии (не знаю, нужен ли, тупо скопипастил)
\usepackage{color}
\usepackage{graphicx}			% чтоб картинки вставлять
\usepackage{amsmath}					% для формул
\usepackage{amsfonts}
\usepackage{amssymb}
\usepackage{listings}                   % для вставок кода
\usepackage{hyperref}                   % оформление ссылок
\usepackage{blindtext}

\graphicspath{ {pic} }       % путь до картинок

%\color[named]{BrickRed}
%\pagecolor[named]{Green}

\begin{document}

\begin{titlepage}
\title{Построение собственного USB устройства ввода в Linux, имея только голый чип}
%\thanks{Version 1.0}
\author{Михаил Белкин}
\maketitle
\end{titlepage}

\tableofcontents
\newpage

\section{Схемотехника.}
\subsection{Выбор редактора.}
    Разработка схемы производится в KiCAD, это очень легковесный и компактный
    opensource редактор. Но при этом несмотря на его внешнюю простоту редактор
    как будто бы кричит нам "Я ничем не хуже чем этот ваш Altium и уж тем более
    Eagle". Поддерживается редактирование многослойных плат, так же используется
    профессиональный подход, при котором схема устройства и печатная плата
    редактируются отдельно. Так же он очень нетребователен к ресурсам
    компьютера.\\
\includegraphics[width=10cm]{kicad1.png}\\
    К редактору имеется собственная библиотека компонентов, которая включает в
    себя компоненты всех популярных производителей. Но если вы принесли
    с китайского базара какую то экзотику, компонент придется разработать
    самостоятельно.\\
    Вывод шаблона печатной платы возможен во всех удобный форматах, включая pdf.
    Так же и Gerber и векторный svg, последнее очень удобно для печати шаблона
    на принтере. Единственное неудобство это не возможность сразу выводить схему
    в растровом формате, приходится самостоятельно конвертировать из svg.
\subsection{Установка редактора и библиотек.}
    Установка редактора в Ubuntu Linux производится очень просто, имеется
    отдельный ppa репозиторий с последней стабильной версией. В то время как
    наиболее полный набор библиотек компонентов можно скачать с гитхаба.\\
    Набор команд для установки KiCAD:
\lstset{language=bash}           % Задаем язык исходного кода
\begin{lstlisting}
sudo add-apt-repository --yes ppa:kicad/kicad-5.1-releases
sudo apt update
sudo apt install --install-recommends kicad
\end{lstlisting}
    Надеюсь устанавливать git вы умеете и про команду git clone вы тоже знаете.
    Вот ссылки на репозитории с библиотеками компонентов KiCAD:\\
    \url{https://github.com/KiCad/kicad-library}\\
    \url{https://github.com/KiCad/kicad-footprints}\\
    \url{https://github.com/KiCad/kicad-symbols}\\
    \url{https://github.com/KiCad/kicad-packages3D}\\
    Установив редактор и добавив библиотеки вы сможете открыть проект со схемой
    устройства
    \href{https://github.com/dltech/usb_device3/tree/main/sch}{по аресу}
\subsection{Выбор элементной базы.}
\subsubsection{Микроконтроллер}
    Для наиболее аккуратной реализации нужен современный микроконтроллер (МК) с
    полноценным аппаратным USB. Совершенно понятно, что таким микроконтроллером
    окажетcя STM32F103C8T6. Мощное ядро ARM Cortex-M3 с их фирменным вложенным
    контроллером прерываний (NVIC) позволит с легкостью справиться с любой
    задачей. А с такой простой как USB геймпад уж тем более. На борту имеется
    64 килобайта FLASH и 20 килобайт SRAM. И этого настолько много, что можно
    вовсе не думать об оптимизации. Теперь о стоимости, когда то я покупал такой
    за 60 рублей, сейчас цена приблизилась к 200, что по прежнему сравнимо по
    стоимости с остальными морально устаревшими микроконтроллерами. Так же в
    пользу данного микроконтроллера говорит наличие подробной
    документации. О том, почему именно F103, тут все просто, это самый дешевый
    МК с USB из тех что может предложить компания ST microelectronics.
\subsubsection{Стабилизатор}
    В шине USB, как известно, 5В, а номинальное напряжение питания МК 3.3В.
    Поэтому необходим понижающий стабилизатор напряжения. Я рассматривал три
    марки стабилизаторов. Они приведены в таблице ниже:
\begin{center}
  \begin{tabular}{ | l | l | l | l | l | }
    \hline
    марка & $U_{in max}$ & корпус & производитель & особенности \\ \hline
    L78L33 & 30 & SOT-89 & ST microelectronics &  \\ \hline
    AMS1117-3.3 & 15 & SOT-223 & AMS semitech & термозащита \\ \hline
    XC6206-33 & 7 & SOT-23 & TOREX & CMOS \\ \hline
     \end{tabular}
\end{center}
    И если первый давно знаком многим радиолюбителям. То последние два это
    стабилизаторы от китайских производителей, которые появились недавно.
    В целом гораздо больше доверия к старому 78l, как минимум из за его
    большого входного напряжения. К тому же AMS1117 мне
    попадались нерабочими, и очень легко пробивались от скачков напряжения,
    не спасая нагрузку. Но хотелось бы компактней и подешевле, к тому же
    компьютер сам по себе стабильный источник питания. Поэтому я выбрал XC6206.
    Довольно необычный новодел на полевых транзисторах, в то время как другие
    два на биполярных. Ниже приведена его структурная схема, на которой видны
    еще и защитные антистатические стабилитроны.\\
\includegraphics[width=10cm]{6206.png}
\subsubsection{Мелочь}
    На форумах можно услышать совет не ставить кварц в целях экономии. Я же
    советую ставить кварц всегда, когда имеем дело с асинхронной шиной для
    стабильной работы устройства. Микроконтроллер можно настроить на работу от
    самого распространенного кварца на 8мГц. Не удивительно что на алиэкспресс
    сразу же нашелся не только планарный, но и очень компактный вариант.
    Размером как 1206 чип резистор. \\
    Резисторы размером 0402 я успел заказать заранее. А вот шунтирующие
    конденсаторы пришлось выпаивать с донорских плат, потому они не такие
    компактные, как хотелось бы (0805). \\
    Расчетный ток потребления десятки миллиампер, потому для стабилизации
    питания хватит и чип керамики, благо такая есть даже на 10 мкФ.
    На всякий случай установлю токоограничивающий резистор по питанию.
    Основная его цель обезопасить компьютер от случайного короткого замыкания.
    Хотя в дорогих флешках на его месте можно встретить чип предохранитель.
    Продолжу экономить и на разъемах, попросту ограничусь площадками под
    проводки.

\subsection{Особенность схемотехники USB.}
    На сайте можно скачать целый документ, посвященный вопросу распайки USB
    разъема. Основной вопрос заключается в возможности программного отключения
    устройства от ПК. Моё же устройство будет всегда включено, поэтому
    подтяжка линии DP к питанию будет постоянной, и осуществляться резистором,
    а не управляться транзистором и портом микроконтроллера.
    Также важно не забыть про защитные резисторы. Провод у меня используется
    готовый от клавиатуры, потому разъем на плате не нужен.

\subsection{Схема устройства.}
    А вот и схема целиком, как видите, все шины питания подключены и
    заземлены защитными конденсаторами. Кварц подключен верно, так же подтянут
    к земле и порт сброса, все как советует официальная документация.
    Кнопки джойстика, как видно, подключены к портам напрямую, т.к. внутри МК
    уже имеются резисторы подтяжки к 3.3В. Так же не обошлось и без так
    называемого ''грязного хака``, для упрощения трассировки платы один из
    портов ввода-вывода использован как вывод земли. Но в этом нет плохого,
    ведь порты после сброса находятся в состоянии с высоким входным
    сопротивлением.\\
    Так же на схеме вы можете увидеть и стандартный разъем SWD для прошивки
    и отладки ПО.\\
\includegraphics[width=15cm]{sch.png}\\


\section{Программное обеспечение.}

\subsection{Тулчейн.}

\subsection{Базовые библиотеки для микроконтроллера.}

\subsection{Структура проекта.}

\subsection{Опрос кнопок.}

\subsection{Периферия USB.}

\subsection{Дескрипоры USB.}

\subsection{Стандартный протокол USB.}

\subsection{USB HID}

\subsection{Отладка.}

%\begin{thebibliography}{9}

%\bibitem{lamport94}
%  Leslie Lamport,
%  Addison Wesley, Massachusetts,
%  2nd edition,
%  1994.
%\end{thebibliography}

%\begin{list}
%    \item 78L33
%    \item AMS1117-3.3
%    \item XC6206-33
%\end{list}

\end{document}
